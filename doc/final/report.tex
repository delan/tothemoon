\documentclass[a4paper,12pt]{article}
\usepackage[utf8]{inputenc}
\usepackage[T1]{fontenc}
\usepackage{parskip}
\usepackage{graphicx}
\usepackage{hyperref}
\usepackage[usenames,dvipsnames]{color}
\hypersetup{
	hidelinks,
	pdfauthor=Delan Azabani; Callum Boyd; Jason Giancono; Robert Lai;
	          Luke Mercuri; Jasmine Quek; Kye Russell,
	pdftitle=Project Design and Management 300: FreedomSpace
}
\let\stdsection\section
\renewcommand\section{\newpage\stdsection}
\makeatletter
\renewcommand\maketitle{
	\begin{titlepage}
		\setcounter{page}{0}
		\centering
		\vspace*{40 mm}
		\huge
		\@title

		\vspace*{10 mm}
		\large
		\@author

		\vspace*{10 mm}
		\@date
	\end{titlepage}
	\pagenumbering{roman}
	\tableofcontents
	\newpage
	\pagenumbering{arabic}
}
\makeatother

\title{Project Design and Management 300\\FreedomSpace}
\date{June 1, 2014}
\author{
	To the Moon \\
	\hspace{10 mm} \\
	Delan Azabani \\
	Callum Boyd \\
	Jason Giancono \\
	Robert Lai \\
	Luke Mercuri \\
	Jasmine Quek \\
	Kye Russell \\
}

\begin{document}

\maketitle

\section{(TODO: Jason) Abstract}

\section{Introduction and objectives}

The use of computers and information technology has increasingly become an
integral factor in research at universities. While the technology and its role
has increased, Curtin University still uses traditional methods such as paper
forms for allocating storage for research projects. In order for us to keep up
with the increasing demand of information technology services, much of these
traditional methods should be digitised and automated.

In order to eliminate the needless paperwork and bureaucracy, Curtin University
could implement a web-based portal for staff members to request and allocate
storage space for each project. This would reduce time loss in the turnaround
caused by waiting for requests, improve the efficacy of record maintenance and
offer extra opportunities for Curtin University to analyse and optimise its
data storage infrastructure.

\subsection{Objectives}

The objective of this project is to create a prototype of a web-based portal
for managing storage space requests at Curtin University. This will allow us to
demonstrate the concept to the stakeholders and refine it as needed. The
prototype isn't required to have the complete functionality, however the user
experience should be complete. The prototype should focus on user experience,
with usability being the main measure of success, as many staff members may not
be proficient with the use of technology.

\subsection{Scope}

The scope of the project will include gathering the user requirements of the
system, designing the user experience of the system and implementing a
prototype of this user experience. Actual functionality and requirements
regarding integration with Curtin University's information systems is outside
the scope of this project, and will depend on the reception of the prototype by
Curtin University's information technology department.

\subsection{Limitations}

The project has not been provided with a budget; all software dependencies must
be freely available, and all hardware must be sourced from group members and/or
computing resources provided by Curtin University.

The prototype will be developed on academic versions of the Visual Studio 2013
Professional integrated development environment, which means that the code
compiled with these systems may not be deployed commercially.

The developers will not have access to Curtin University's internal LDAP server
infrastructure and other information systems. This means that the prototype's
main focus will be on the user experience and the backend which facilitates
that. Enhancements such as integration with learning management systems and
directory services will be features likely available in a final product but
will not be represented in the prototype.

There are only twelve weeks to complete the entire project, including the
solicitation of requirements and user testing. Developers will also provide
different sets of skills, while being restricted by different unit timetables.

\subsection{Approach}

Our team is developing the prototype using the Scrum methodology. First the
Product Backlog is decided upon --- each item is allocated a priority, then
a subset of the user stories are chosen to be completely designed, implemented
and tested for each sprint. The team meets before and after each sprint to
discuss how the sprint went, as well as what needs to be completed in the next
sprint. This allows us to efficiently adapt to any issues with our
requirements.

\subsection{Structure}

This report will guide us through the background which has lead to this
project, the requirements which have been decided upon for the prototype, the
design of the web-based portal and the process used during development of this
project and prototype.

\section{(TODO: Jasmine) Background}

\subsection{Initial project management approach}

\subsection{Tools considered for development}

\subsection{User interface mockups}

\section{(TODO: Callum) Product backlog}

\section{Sprint planning}

\subsection{Planning for sprint 1}

The user stories that were chosen for the first sprint were the user stories
with the greatest importance, serving as the metaphorical foundations of the
project. To do this we rearranced and categorised the user stories by their
importance, access level requirements, and dependencies to yield the tasks
which required the lowest level of requirements to complete first. The first
sprint should contain everything that would be mandatory in the second and
third sprints, namely:

\begin{itemize}
	\item As an administrator, I want to log in so that I can review and
	      approve requests;
	\item As a principal investigator, I want to submit storage requests so
	      that I can have access to a storage space for a research project;
	\item As a user, I want to log in so that I can perform appropriate
	      functions applicable to my role;
	\item As a user, I want to view a list of my storage spaces so that I
	      know my access level for my research projects;
	\item As an administrator, I want to approve a request so that storage
	      can be provisioned, expanded or access permissions changed;
	\item As an approver, I want to log in so that I can approve storage
	      requests; and
	\item As an approver, I want to review and approve a storage request so
	      that storage can be provisioned.
\end{itemize}

These user stories are what we collectively prioritised as the most important
tasks to complete before any other parts of the project can be undertaken.

The breakdown that we had planned for the project as a whole was to correlate
all user stories to a number depending on what required the least functionality
as well as which user stories required others to be completed before they could
be implemented. By doing this, we managed to organise all of the user stories
into three groups, which ideally would be allocated to each sprint. The first
sprint would involve everything on the lowest level, which includes logging
into the website, viewing spaces and reviewing requests as the users with the
highest levels of access permissions.

With this we estimated that it would take approximately two weeks. This would
give us enough time for the group to orientate and establish suitable
communications between all members. After that, we would be able to work
smoothly throughout the project. Within this fortnight, one week will be set
aside to create the user login system and have it operational, then three days
will be allocated to implement the viewing of lists and reviewing of storage
requests. The remaining four days would go towards fixing defects and polishing
what has been done thus far, in addition to working on other projects.

\subsection{Planning for sprint 2}

What was designated for the second sprint were the second set of user stories,
which involved building up from what should have been accomplished in the first
sprint. Ideally, this would mean that authentication and request approvals
would have been completed.

The second set of user stories would involve creating a access permissions
system whereby --- depending on a particular user's access level --- the system
would provide and/or restrict functionality. The user stories that were chosen
for the second sprint include:

\begin{itemize}
	\item As a data manager, I want to grant other researchers access to my
	      storage space so that I can collaborate with others;
	\item As a data manager, I want to submit additional storage requests
	      so that I can store more research data in my storage space;
	\item As a principal investigator, I want to grant other researchers
	      access to my storage space so that I can collaborate with others;
	\item As a principal investigator, I want to grant other researchers
	      manager level access for my storage space, so that they can
	      submit administrative requests;
	\item As a principal investigator, I want to submit additional storage
	      requests so that I can store more research data in my storage
	      space;
	\item As an administrator, I want to change the principal investigator
	      for a storage space to replace the existing principal
	      investigator; and
	\item As an approver, I want to view a list of all storage requests so
	      that I am aware of all requests by principal investigators in my
	      faculty.
\end{itemize}

Assuming the aforementioned user stories are completed by the end of the
sprint, the project should also be able to handle the next set of user stories
to be implemented. This is because the second sprint is purely intended to
ensure that access levels and data storage function correctly.

From the user stories outlined above, the second level of the breakdown is
found, including the tasks that require logging in as a user with a specific
user access level. This involves submitting and granting storage space requests
for `middle level' users.

The time estimated for this sprint would again be two weeks, with the granting
of storage requests taking one week. The additions to the prototype that handle
request submission should only take three days at most to implement, and the
remaining days would be allocated to the ability to view and change the
principal investigator of storage spaces.

\subsection{Planning for sprint 3}

By the third sprint hopefully, the previous user stories would have been
completed and implemented correctly. The rest of the user stories are small
additions onto the already existing system, so even though there are a vast
quantity more user stories in the sprint than others, these user stories all
function in a similar fashion to one another, so implementation would require
one function each, plus significant reusability for the other user stories that
require that functionality.

The user stories selected to be implemented in the third sprint were:

\begin{itemize}
	\item As a data manager, I want to receive email notifications so that
	      I am aware that requests have been completed;
	\item As a data manager, I want to revoke existing researchers' access
	      to my storage space so that they can't access my data;
	\item As a data manager, I want to view the status of my request so
	      that I know its progress;
	\item As a principal investigator, I want to receive email
	      notifications so that I am aware that requests have been
	      completed;
	\item As a principal investigator, I want to revoke an existing
	      researcher's manager level access for my storage space so that
	      they cannot submit administrative requests;
	\item As a principal investigator, I want to revoke existing
	      researchers' access to my storage space so that they can't access
	      my data;
	\item As a principal investigator, I want to view the status of my
	      request so that I know its progress;
	\item As a researcher, I want to receive email notification when I am
	      granted access to a storage space so that I can access the data;
	\item As a researcher, I want to receive email notification when my
	      access to a storage space is revoked so that I am aware I can no
	      longer access it;
	\item As an administrator, I want to add a comment to a requests so
	      that I can highlight any special requirements or additional
	      information related to the request;
	\item As an administrator, I want to send appropriate notifications so
	      that users can start using their storage spaces;
	\item As an administrator, I want to view a list of requests by
	      submission date so that I can review them chronologically; and
	\item As an approver, I want to view a list of all storage spaces for
	      my faculty so that I can determine storage space allocated for my
	      faculty.
\end{itemize}

This would be the last level of the breakdown, where all of the more
superficial additions are implemented. As a team, the decision was made that
because these user stories did not involve \textit{changing} access levels, but
still relied on the \textit{existence} of access levels, they would be easier
to implement.

The time available would be three weeks, and this would include finishing the
formatting and patching all remaining defects, as well as ensuring that the
system runs in accordance with the client's requirements.

The email notification module would require three days, and viewing the status
of requests would require a further four days to implement. The second week
would be dedicated to revocation of storage space access, as this involves
changing the access control data between a user and a space, rather than the
access role of a user.

\section{Scrum meetings}

\subsection{Scrum meeting 1}

Date: 2014-04-03

Start time: 4:00 p.m.

Discussing how the project should be completed and the technologies we are
using to implement the prototype.

\begin{itemize}
	\item Front end: Bootstrap;
	\item Back end: ASP.NET MVC 5;
	\item Database: Microsoft SQL Server;
	\item Time logging: Google Docs form;
	\item Number of sprints: three.
\end{itemize}

Front end tasks to complete during the first sprint:

\begin{itemize}
	\item Requests pages and forms;
	\item Login screen; and
	\item Notification space.
\end{itemize}

End time: 5:00 p.m.

\newpage

\subsection{Scrum meeting 2}

Date: 2014-04-15

Start time: 12:10 p.m.

Meeting for discussing what is functioning in the first sprint, and what needs
to be added during sprint two.

The list of user stories that were completed were:

\begin{itemize}
	\item Submission of storage requests; and
	\item Users being able to log in.
\end{itemize}

Tasks that were partially completed were:

\begin{itemize}
	\item Administrators being able to approve and change permissions; and
	\item Principal investigators submitting storage requests for research.
\end{itemize}

All of the user stories were completed on the front end load where users could
see what it looked like, but all of the functionality was yet to be finished.
These will need to be completed within the next sprint because there were
issues with access levels of the different types of users, and we were unsure
how to handle them until we completely designed the access levels for all the
users, which would be done in the second sprint.

End time: 1:00 p.m.

\newpage

\subsection{Scrum meeting 3}

Date: 2014-04-25

Start time: 1:03 p.m.

Task allocation:

\begin{itemize}
	\item Robert: UML diagram, user stories, list of back end and front end
	      design decisions, implementation and sprint review;
	\item Jasmine: high level task breakdown, effort estimation;
	\item Callum: implementation of approvers' user stories;
	\item Delan and Luke: implementation of granting and revocation of
	      permissions for data managers;
	\item Kye: implementation of user authentication and access levels; and
	\item Jason: implementation of the permissions list view.
\end{itemize}

All further issues will be raised on the Facebook group, so that all members
can have the opportunity to assist. The original plan from earlier sprints was
too complicated, and it will be modified to make it simpler. Approvers will be
managed with a flag at the user level, indicating whether they are allowed to
approve requests or not.

The UML diagram was designed for the implementation.

End time: 2:34 p.m.

\newpage

\subsection{Scrum meeting 4}

Date: 2014-05-01

Start time: 4:00 p.m.

Meeting for discussing what is functioning in sprint two and what needs to be
added to sprint three, as well as allocation of report writing tasks to group
members. The user stories that were planned for full implementation were for
data managers and principal investigators to be able to submit requests, and to
allow privileged users to grant and revoke access to storage spaces.

The list of user stories that were completed were:

\begin{itemize}
	\item Submission of storage requests for data managers and principal
	      investigators;
	\item Viewing a list of storage spaces for approvers;
	\item Administrators being able to approve and change permissions; and
	\item Principal investigators submitting storage requests for research.
\end{itemize}

User stories that were partially completed were:

\begin{itemize}
	\item Administrators being able to change the principal investigator on
	      a storage space; and
	\item Principal investigators granting researchers read-only, writable
	      and data manager level access to storage spaces.
\end{itemize}

As a team, the decision was made to shift a majority of the focus towards the
documentation of the sprints and the report. The report will be created for
global editing within the members to update and edit where possible. We will
finish the demonstration for displaying research by the next week. Jasmine
has been assigned to conduct the final presentation. Callum and Robert have
been assigned to write the `details of design' and `Scrum meeting' sections of
the final report. Luke and Delan were assigned to completing the backend models
and controllers, while Jason and Kye are to finish developing the views and the
user interface. Callum shall polish the finish ed project for presentation and
any additional documentation within the report.

End time: 4:37 p.m.

\subsection{Scrum meeting 5}

Date: 2014-05-15

Start time: 3:59 p.m.

By the end of sprint three, we had assumed that everything for the project
would be functional, except for some bugs which are to be expected.

The list of user stories that were completed were:

\begin{itemize}
	\item Administrators being able to change the principal investigator
	      for a storage space;
	\item Principal investigators granting researchers read-only, writable
	      and data manager level access to storage spaces;
	\item Reception of email notifications as any privileged user;
	\item Revocation of access as any user;
	\item Addition of comments to requests;
	\item Viewing a list of request sorted by submission date; and
	\item Viewing a list of storage spaces for a specific faculty.
\end{itemize}

Tasks that were partially completed were:

\begin{itemize}
	\item Formatting for the user interface.
\end{itemize}

Report task reallocation:

\begin{itemize}
	\item Callum: product backlog;
	\item Delan: details of design, conclusion and summary;
	\item Jasmine: background, implementation details;
	\item Jason: abstract, introduction and objective;
	\item Kye: project review;
	\item Luke: sprint review and retrospective; and
	\item Robert: sprint documentation and planning.
\end{itemize}

Callum and Jasmine will edit and correct the final report copy, while Delan
will complete the formatting and typesetting of the report.

End time: 4:24 p.m.

\subsection{Scrum meeting 6}

Date: 2014-05-26

Start time: 9:11 a.m.

Finalisation of report, documentation and presentation.

The group shall have the report finished by Friday for editing and formatting.
Delan will create the PowerPoint file for the final presentation. Depending on
Aneesh's answer, everyone could be presenting, or only some group members.

End time: 9:21 a.m.

\subsection{Scrum meeting 7}

Date: 2014-05-29

Start time: 4:06 p.m.

Last meeting before the final presentation.

Contact Delan when finishing report segments so formatting can be completed.
Jasmine is unwell; if she is unable to present, duties will be passed onto
Callum, Robert or Luke depending on the day of the presentation. Small changes
will be made to the prototype.

End time: 4:27 p.m.

\section{(TODO: Delan) Details of the design}

\section{(TODO: Jasmine) Details of the implementation}

\section{Sprint reviews and retrospectives}

\subsection{Review of sprint 1}

In the first sprint, the user stories decided upon were fairly ambitious. As a
group we completed a number of allocated user stories including:

\begin{itemize}
	\item As an administrator, I want to log in so that I can review and
	      approve requests;
	\item As a user, I want to log in so that I can perform appropriate
	      functions applicable to my role; and
	\item As an approver, I want to log in so that I can approve storage
	      requests.
\end{itemize}

These user stories overlapped in functionality and so were considered as one
single task, which was completed by Delan.

\begin{itemize}
	\item As a principal investigator, I want to submit storage requests so
	      that I can have access to storage space for research projects;
	\item As a data manager, I want to submit additional storage requests
	      so that I can store more research data in my storage space; and
	\item As a principal investigator, I want to submit additional storage
	      requests so that I can store more research data in my storage
	      space.
\end{itemize}

As with the login user stories, the above stories were considered to be one
single task due to the similarity in their functionality, and they were
completed by Kye.

\begin{itemize}
	\item As a user, I want to view a list of my storage spaces so that I
	      know my access level for my research projects; and
	\item As an approver, I want to view a list of all storage spaces for
	      my faculty so that I can determine storage space allocated to my
	      faculty.
\end{itemize}

This task was not completed in its entirety at this stage but the majority of
it had been completed and was ready for presentation. Completion of this task
was a joint effort between Jason and Kye.

There were also planned user stories that at this point were not yet complete,
such as:

\begin{itemize}
	\item As an administrator, I want to approve a request so that storage
	      can be provisioned, expanded or access permissions changed; and
	\item As an approver, I want to review and approve a storage request so
	      that storage can be provisioned.
\end{itemize}

Unfortunately, due to a combination of insufficient time management, informal
role allocation and a general team-wide unfamiliarity with the chosen
development environment, these user stories were not addressed.

Finally, there existed additional items completed concerning improvements to
user experience that were not initially addressed in meetings prior to the
commencement of the sprint, such as:

\begin{itemize}
	\item Redesign of navigation bar; and
	\item Front end addition of interleaved links to relevant pages.
\end{itemize}

These changes were completed by Luke.

\subsection{Retrospective on sprint 1}

This sprint was objectively rather productive. Of ten assigned tasks, the group
worked together to complete eight requirements that were initially planned, as
well as the addition of two unplanned tasks. These additions were vital in
order to ensure easy navigation of the project by the end user.

Issues encountered mainly concerned:

\begin{itemize}
	\item The lack of a clear definition of group member task allocations.
	\\\\  This was the most fundamental flaw with our approach to this
	      sprint. Due to a combination of poor communication and job
	      allocation, we had a situation where many group members had
	      accidentally begun work on the same tasks and for this reason,
	      tasks were left untouched while a lot of effort was duplicated
	      and ultimately discarded.
	\item Many team members at this stage were completely unfamiliar with
	      the chosen working environment of ASP.NET MVC 5.
	\\\\  While not quite as important as the previous issue, at this
	      stage, it was definitely a hindrance to be trying to write code
	      in an environment that none of the group members were familiar
	      with.
\end{itemize}

For these reasons, the group as a whole resolved to be more explicit in task
allocations and ownership. Expected task time frames were decided on moving
forward, as a preventative measure for tasks being left incomplete.

\subsection{Review of sprint 2}

At the epoch of the second sprint, user stories decided on were scaled back
quite drastically to accommodate incomplete items in the sprint backlog. Of the
chosen user stories, the following were completed:

At this stage in the project, there were a worrying number of incomplete user
stories now moved into our unfinished item backlog, such as:

\begin{itemize}
	\item As an approver, I want to review and approve a storage request so
	      that storage can be provisioned;
	\item As an approver, I want to view a list of all storage requests so
	      that I am aware of all requests by principal investigators in my
	      faculty;
	\item As a data manager, I want to grant other researchers access to my
	      storage space so that I can collaborate with others;
	\item As an administrator, I want to approve a request so that storage
	      can be provisioned, expanded or access permissions changed; and
	\item As an approver, I want to review and approve a storage request so
	      that storage can be provisioned.
\end{itemize}

Of these, the most important functionality by far were the user access levels,
and to not yet have this complete took a very noticeable cleave out of the
team's morale.

In addition to the planned functionality to be worked on in this sprint, there
were also a number of features added supplementary to what was planned:

\begin{itemize}
	\item Dashboard front end using static HTML.
\end{itemize}

As completed by Jasmine, this was an invaluable feature to have going into the
presentation for the second sprint as it was the first major step towards
transposing our initial mockups and planning into an actual product.

\begin{itemize}
	\item Users being able to cancel/delete their own space requests; and
	\item Requests now storing a timestamp to reflect when they were made.
\end{itemize}

These functionalities were implemented by Kye.

\subsection{Retrospective on sprint 2}

Aspects that worked well at this point were unfortunately rather scarce.

Over the course of the sprint, many obstacles became apparent:

\begin{itemize}
	\item Members were sick due to the time of the year.
	\\\\  While it is unavoidable that this can happen, the team
	      unfortunately had made no provisions to handle the situation.
	      When people were unable to complete their tasks, there was little
	      and rather poor communication within the group to attempt to
	      overcome the issue.
	\item Sprint backlog items were not adequately addressed.
	\\\\  Over the course of this sprint, the team morale reached a
	      global minimum. Consequently team members felt uncompelled to
	      address any tasks wihch resulted in a less than desirable yield.
\end{itemize}

After what was undeniably the worst sprint so far, the team as a whole devised
stricter guidelines for the next sprint in order to produce more consistent
results going forward. Guidelines considered of utmost importance were:

\begin{itemize}
	\item Motivate team members more effectively to stick to designated
	      tasks and time schedules; and
	\item Issues that arise will be brought up both in person as well as
	      the Facebook group in order to alert all group members of
	      potential problems as soon as they develop.
\end{itemize}

\subsection{Review of sprint 3}

It was near the end of the second sprint where the team didn't realise how
close the dealine was to the final presentation of the project, and it was only
then did the group finally shift gears and come together to start finishing off
the user stories rather quickly. With efforts from Kye and Luke, the user
access levels were quickly finished and with that, the following list of user
stories were also completed with it, from the backlog of sprint two:

\begin{itemize}
	\item As an approver, I want to review and approve a storage request so
	      that storage can be provisioned;
	\item As an approver, I want to view a list of all storage requests so
	      that I am aware of all requests by principal investigators in my
	      faculty;
	\item As a data manager, I want to grant other researchers access to my
	      storage spaces so that I can collaborate with others;
	\item As an administrator, I want to approve a request so that storage
	      can be provisioned, expanded or access permissions changed; and
	\item As an approver, I want to review and approve a storage request so
	      that storage can be provisioned.
\end{itemize}

When this was completed, Jason in adition to Luke and Kye worked tirelessly to
get the project back onto a reasonable pcae, and could possibly complete it on
time. These were the remaining user stories that required work:

\begin{itemize}
	\item As a data manager, I want to revoke existing researchers' access
	      to my storage spaces so that they can't access my data;
	\item As a data manager, I want to view the status of my request so
	      that I know its progress;
	\item As a principal investigator, I want to receive email
	      notifications so that I am aware that requests have been
	      completed;
	\item As a principal investigator, I want to revoke an existing
	      researcher's manager level access for my storage spaces so that
	      they cannot submit administrative requests;
	\item As a principal investigator, I want to revoke existing
	      researchers' access to my storage spaces so that they can't
	      access my data;
	\item As a principal investigator, I want to view the status of my
	      requests so that I know its progress;
	\item As an administrator, I want to send appropriate notifications so
	      that users can start using their storage spaces;
	\item As an administrator, I want to view a list of requests by
	      submission date so that I can review them chronologically; and
	\item As an approver, I want to view a list of all storage spaces for
	      my faculty so that I can determine storage space allocated for
	      my faculty.
\end{itemize}

With this happening, Jasmine and Callum were cleaning and polishing the project
so that for the final presentation it would work as intended. At the end of
this sprint the initial plan was that the project would be complete, and with
a joint effort from all members the project was completed on time.

\subsection{(TODO: Delan) Retrospective on sprint 3}

\section{(TODO: Kye) Project review}

\section{(TODO: Delan) Conclusions and summary}

\section{(TODO: Callum) Appendices}

\section{(TODO: Callum) References}

\end{document}
