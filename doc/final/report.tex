\documentclass[a4paper,titlepage,12pt]{article}
\usepackage[margin=1in]{geometry}
\usepackage{filecontents}
\usepackage[backend=biber,authordate]{biblatex-chicago}
\addbibresource{report.bib}
\usepackage{parskip}
\usepackage{graphicx}
\usepackage{listings}
\lstset{basicstyle=\ttfamily, basewidth=0.5em}
\usepackage{hyperref}
\hypersetup{
	hidelinks,
	pdfauthor=Delan Azabani; Callum Boyd; Jason Giancono; Robert Lai;
	          Luke Mercuri; Jasmine Quek; Kye Russell,
	pdftitle=Project Design and Management 300: FreedomSpace
}
\usepackage{pdflscape}
\usepackage{textpos}

\setlength{\TPHorizModule}{1mm}
\setlength{\TPVertModule}{1mm}
\setlength{\abovecaptionskip}{15pt plus 3pt minus 2pt}
\setlength{\belowcaptionskip}{15pt plus 3pt minus 2pt}

\let\stdsection\section
\renewcommand\section{\newpage\stdsection}

\newcommand\figimg[4][\textwidth]{
	\begin{figure}
		\caption{#4}
		\label{fig:#2}
		\includegraphics[width=#1]{#3.png}
	\end{figure}
}

\title{Project Design and Management 300\\FreedomSpace}
\date{June 1, 2014}
\author{
	To the Moon \\
	\hspace{10 mm} \\
	Delan Azabani \\
	Callum Boyd \\
	Jason Giancono \\
	Robert Lai \\
	Luke Mercuri \\
	Jasmine Quek \\
	Kye Russell \\
}

\pagenumbering{gobble}

\begin{document}

\maketitle
\pagenumbering{roman}
\tableofcontents
\listoffigures
\pagenumbering{arabic}

\section{(TODO: Jason) Abstract}

\section{Introduction and objectives}

The use of computers and information technology has increasingly become an
integral factor in research at universities. While the technology and its role
has increased, Curtin University still uses traditional methods such as paper
forms for allocating storage for research projects. In order for us to keep up
with the increasing demand of information technology services, many of these
methods should be digitised and automated.

In order to eliminate this needless paperwork and bureaucracy, Curtin
University could implement a web-based portal for staff members to request and
allocate storage space for each project. This would reduce time loss in the
turnaround caused by waiting for requests, improve the efficacy of record
maintenance and offer extra opportunities for Curtin University to analyse and
optimise its data storage infrastructure.

\subsection{Objectives}

The objective of this project is to create a prototype of a web-based portal
for managing storage space requests at Curtin University. This will allow us to
demonstrate the concept to the stakeholders and refine it as needed. The
prototype isn't required to have the complete functionality, however the user
experience should be complete. The prototype should focus on user experience,
with usability being the main measure of success, as many staff members may not
be proficient with the use of technology.

\subsection{Scope}

The scope of the project will include gathering the user requirements of the
system, designing the user experience of the system and implementing a
prototype of this user experience. Actual functionality and requirements
regarding integration with Curtin University's information systems is outside
the scope of this project, and will depend on the reception of the prototype by
Curtin University's information technology department.

\newpage

\subsection{Limitations}

The project has not been provided with a budget; all software dependencies must
be freely available, and all hardware must be sourced from group members and/or
computing resources provided by Curtin University.

The prototype will be developed on academic versions of the Visual Studio 2013
Professional integrated development environment, which means that no part of
this prototype may be deployed commercially.

The developers will not have access to Curtin University's LDAP directory
service and other information systems. This means that the prototype's main
focus will be on the user experience and the backend which facilitates that.
Enhancements such as integration with learning management systems and directory
services will be features likely available in a final product but will not be
represented in the prototype.

There are only twelve weeks to complete the entire project, including the
solicitation of requirements and user testing. Developers will also provide
different sets of skills, while being restricted by different unit timetables.

\subsection{Approach}

Our team is developing the prototype using the Scrum methodology. First the
Product Backlog is decided upon --- each item is allocated a priority, then
a subset of the user stories are chosen to be completely designed, implemented
and tested for each sprint. The team meets before and after each sprint to
discuss how the sprint went, as well as what needs to be completed in the next
sprint. This allows us to efficiently adapt to any issues with our
requirements.

\subsection{Structure}

This report will guide us through the background which has led to this project,
the requirements which have been decided upon for the prototype, the design of
the web-based portal and the process used during development of this project.

\section{Background}

\subsection{Initial project management approach}

Our solution approach was to first prioritise the user stories given, so that
vital functionality would be implemented in early sprints. Shortly afterwards,
the team had a meeting to discuss the various permissions of the different user
account roles. In this meeting we made a permissions table, as shown in
Figure~\ref{fig:permissions}, so as to clear up any confusion surrounding the
wide variety of user privileges.

The overarching method was to figure out the solutions for the larger problems
as a group, and then to allocate tasks to be solved by assorted group members
by the end of the next sprint. Existing free technologies were used to
coordinate project management and communications, including a Facebook group,
Google form for time logging, Trello to organise user stories as well as
Basecamp. The time allocated to various tasks in the project can be seen in the
Gantt chart in Figure~\ref{fig:gantt}.

\begin{figure}[h]
	\caption{The user access permissions model for this prototype}
	\label{fig:permissions}
\begin{lstlisting}
                Receive approval notifications
                | View available spaces
                | | Request access to space
                | | | Grant/revoke access to space
                | | | | Promote/demote Data Manager
                | | | | | Request additional space
                | | | | | | Approve new/expanded space
                | | | | | | | Set Pri.Inv. owning space
                | | | | | | | | Comment on requests
                | | | | | | | | |
        Admins  X X X X X X X X X
     Approvers  X X X X X X X X
 Data Managers  X X X X X X
   Researchers  X X X
\end{lstlisting}
\end{figure}

\begin{landscape}
	\quad % ugly hack because there needs to be some text here
	\thispagestyle{empty}
	\begin{textblock}{270}(-12,0)
		\figimg[270mm]{gantt}{gantt}
			{The Gantt chart for this project}
	\end{textblock}
\end{landscape}

\subsection{Tools considered for development}

\begin{itemize}
	\item Django is a `high-level Python Web framework' that adheres to the
	      model-view-controller pattern.
	      (\cite{Django2014})
	\item XAMPP is an `Apache distribution containing MySQL, PHP, and Perl'
	      with a unified installation and configuration workflow for
	      Windows, Mac OS X and Linux.
	      (\cite{ApacheFriends2014})
	\item Flask is a `microframework for Python' which provides template
	      rendering and request routing, but avoids enforcing any
	      particular patterns.
	      (\cite{Ronacher2014})
	\item Bootstrap is a frontend framework with HTML, CSS and optionally
	      JavaScript-based `reusable components'.
	      (\cite{Twitter2014})
	\item ASP.NET MVC 5 is a `framework for building scalable,
	      standards-based web applications using well-established design
	      patterns and the power of ASP.NET and the .NET Framework.'
	      (\cite{Microsoft2014})
	\item Basecamp is a project management tool that provides scheduling,
	      task management, milestones and defect tracking.
	      (\cite{37signals2004})
	\item Facebook Groups allow text posts, images, files and questions
	      to be shared privately among a set of users.
	      (\cite{Facebook2014})
	\item Trello is a web-based group project manager that implements the
	      \textit{kanban} paradigm.
	      (\cite{FogCreek2014})
	\item draw.io is a `free online diagram drawing application' which can
	      create UML, entity-relationship, network and many other types
	      of diagrams.
	      (\cite{JGraph2014})
	\item Google Drive consists of cloud storage, live collaborative
	      document editing, and a tool for online forms.
	      (\cite{Google2014})
	\item GitHub provides `powerful collaboration, code review, and code
	      management for open source and private projects.'
	      (\cite{GitHub2014})
\end{itemize}

\newpage

\subsection{Rationale for using our tools}

The tools we decided to use throughout the life of the project are:

\begin{itemize}
	\item Backend: ASP.NET MVC 5;
	\item Frontend: Bootstrap;
	\item Project management: Facebook Groups, Basecamp and Trello;
	\item Version control: GitHub;
	\item Diagrams: draw.io; and
	\item Documentation: Google Docs and \LaTeX.
\end{itemize}

The group encountered difficulty deciding on what framework we would use for
the project, as few of us had much exposure or experience in backend web
development. We decided that it would be voted upon, and as a subset of the
group were all familiar with C\#, we swayed towards ASP.NET MVC 5.

As ASP.NET MVC 5 comes packaged with Bootstrap, it was consequently common
sense to use both alongside each other for the project. The team agreed that
Bootstrap was most suitable, and as a further plus it had stock templates that
would help us in early development stages.

Shortly after a group was formed, as we were all Facebook friends with each
other it seemed almost natural to start a Facebook Group for easy
communications outside of tutorials and seminars. The ability to design polls
for members of the group was a huge plus as it made it a lot easier to organise
meetings, and especially so for our relatively large group of seven people.

A Basecamp project was also set up for the group assignment. We originally
chose to use Basecamp because of its various services such as to-do lists and
file sharing. However, we did not get much of a chance to use it as we did most
task allocation in person during Scrum and team meetings. Additionally, many of
the addons that were desirable for the documentation side of the project were
rather costly and deterred us from using it.

Our Trello board was only set up after the group had task allocation issues as
can be seen in the review of the first sprint. A few members of the group
didn't originally want to use Trello as a project management tool due to seeing
it used to a messy result elsewhere. However after group members were
mistakenly working on the same tasks it was decided it was necessary. We used
one board for the entire project, with user stories as cards, and several lists
depicting the stages of the completeness of those user stories.

Alas, by this point we were using too many project management tools and it was
inconvenient for people to regularly check on several applications, websites
and devices. The group then ended up having the minute taker post all meeting
minutes onto the Facebook Group for easy access. Trello ceased to be used after
the second sprint.

As several group members were familiar with GitHub and were very pleased with
its offerings, it was suggested to be used in the project. GitHub also allows
for several private repositories if signed up with an educational email address
which was another benefit, as we didn't want to risk other groups plagiarising
our hard work.

draw.io was used before the unofficial `sprint zero' presentation to display
our initial mockups of our website. At this point we had not started any
development, and the focus was on planning the first sprint. The group decided
to use Draw.io as a couple of members had used it previously for another group
assignment where various UML diagrams and mockups were drawn. The only downside
is that it does not support online live collaboration, but that wasn't too much
of an issue. This tool was also then used for the final UML and ER diagrams for
the project as documented in this report.

Google Docs was used heavily in the creation of the final report, as well as
communicating between members progress of web backend functionality. A group
member suggested using Google Docs and this was found in agreement with other
members. One of its greatest merits is its ability to support online live
collaboration. When working among a group of seven busy university students,
this support is necessary. This feature was extremely helpful in seeing where
other group members were up to in their allocated section of the report, as
well as being able to refer to others' sections when working on one's own
sections.

The final report was transcribed from a Google Docs file to be typeset using
\LaTeX, in order to exercise a greater degree of control over the aesthetics
and typography, while automating tedious tasks such as figure layout and
bibliography management.

\newpage

\subsection{User interface mockups}

A handful of user interface mockups were created for the initial project
presentation. These were completed before any development of the prototype had
started, and can be seen in Figures \ref{fig:login-mockup},
\ref{fig:admin-mockup}, \ref{fig:researcher-mockup},
\ref{fig:preferences-mockup} and \ref{fig:request-mockup}.

\figimg{login-mockup}{../mockups/login}
	{A mockup for the login screen}

\figimg{admin-mockup}{../mockups/admin}
	{A mockup for the dashboard as seen by an administrator}

\figimg{researcher-mockup}{../mockups/researcher}
	{A mockup for the dashboard as seen by a researcher}

\figimg{preferences-mockup}{../mockups/preferences}
	{A mockup for the preferences screen}

\figimg{request-mockup}{../mockups/request}
	{A mockup for the storage space request form}

\section{Product backlog}

Our first goal with user stories was to divide them into three groups
representing their priority for completion, with `1' being the highest and `3'
being the lowest. In the highest priority group we placed stories that we
believed to be related to core parts of the system. The second group were
stories that enhanced the core functionality of the system and added features
and other useful tools for the users. The lowest priority group were stories
that enhanced ease of access and usability as well as features that were not
core functionality.

In the first priority tier of user stories, we placed any stories relating to
authentication and user access levels. We felt that this was core to the system
because the users would have different permissions and access. There were quite
a few stories relating to logging in and being able to perform their functions
from that. Out of the seven user stories that were decided to be `priority
one', three were relating to login functions.

\begin{itemize}
	\item As an administrator, I want to login so that I can review and
	      approve requests;
	\item As a user, I want to login so that I can perform appropriate
	      functions applicable to my role; and
	\item As an approver, I want to login so that I can approve storage
	      requests.
\end{itemize}

In the first story we understand that the administrators of the system want to
be able to log in and approve requests. From this we can assume that they want
to be able to log in so that they will be able to have access to this feature,
not just anyone.

\figimg{login}{login}{The login screen}

We can further assume that this is also related to wanting some level of system
and user security so that only authorised personnel have access to requests.
This is also seen in the third story where the approver wants to be able to log
in to approve requests. From this we can see that approving requests is a
privileged feature and only these two types of users are able to do this. This
can be seen in the login screen feature shown in Figure~\ref{fig:login}.

The second user story shows that all users do want to be able to log in and
perform functions applicable to their role. This can be expanded to users want
to be able to log in and only see functions related to their own role,
therefore only displaying functions that they can use. From this assumption we
can work out that users only want to see what they can do and they don't want
to see everything else, as illustrated in Figure~\ref{fig:user0} and
Figure~\ref{fig:user2}.

\figimg{user0}{user0}{The dashboard as seen by a plebeian user}

\figimg{user2}{user2}{The dashboard as seen by an administrator}

While this may seem like an ease of use story it is important to make sure that
all users can log in as soon as the first part of the working system is
complete; therefore this was classified as a first priority story.

\begin{itemize}
	\item As a principal investigator, I want to submit a storage request
	      so that I can have access to storage space for a research
	      project;
	\item As a user, I want to view a list of my storage spaces so that I
	      know my access level for my research projects;
	\item As an administrator, I want to approve a request so that storage
	      can be provisioned, expanded or access permissions changed; and
	\item As an approver, I want to review and approve a storage request so
	      that storage can be provisioned.
\end{itemize}

These four stories are the remainder of the tier one user stories. All four of
them relate to storage access; as this system is about managing access to
storage for researchers, it is very important that they are completed first.

The first story is about the principal investigator, which is the role of the
person who wants to have the storage set for their project. From this first
story we can gather that principal investigators want to be able to easily
request storage spaces as this is a key part of their role.

The second story can be split into two parts, the first part being that users
want to be able to see their storage spaces. This is a critical function of the
system because there is no point to having storage spaces if no one can see
them. The second part could very easily be forgotten due to the significance of
the first part and that is that the users also want to know their access levels
for these projects that have space allocated.

The third story gives insight into what is wanted from the system in terms of
changing the storage of users. From this we can understand that administrators
want to be able to change the permissions, expand and approve storage. This
means the system needs to be able to dynamically set storage and user
permissions to add in or remove users' access to storage spaces. This was
classified as a high priority because it is a core part of the system that
access is granted and only to those who are approved.

The fourth story is quite similar to the third in that it is about approving
storage requests, which is why it is grouped with it. However it also helps
define the roles of administrator and approver, because from this story we can
see that only administrators want to be able to change permissions and expand
storage, where approvers only want to be able to approve as their name would
suggest.

What we learned from this first set stories is that we need to have a
well-defined set of rules of user permissions. From the user stories we know
that each user has a role within a project and that while some of their roles
do overlap or some functions they are all different. These user levels and
roles help define what users are allowed to do and access within the system.
Due to the key nature of system integrity we decided to come up with a diagram
of user permission levels for all the user types. This was used throughout the
design phase to create the different user logins in the system.

For the second level of user stories we see many more stories concerning
collaboration and changing user access levels for the existing storage spaces.

\begin{itemize}
	\item As a data manager, I want to grant other researchers access to my
	      storage space so that I can collaborate with others;
	\item As a data manager, I want to submit additional storage requests
	      so that I can store more research data in my storage space;
	\item As a principal investigator, I want to grant other researchers
	      access to my storage space so that I can collaborate with others;
	\item As a principal investigator, I want to grant other researchers
	      data manager level access for my storage space so that they can
	      submit administrative requests; and
	\item As a principal investigator, I want to submit additional storage
	      requests so that I can store more research data in my storage
	      space.
\end{itemize}

From the first two stories we see a new user role, the data manager, which is a
user who controls the data and access of the projects. From the first of these
stories we can see that the data managers need to be able to control who has
access to their storage spaces which they want to use for collaboration. We
also see that in the third story the principal investigators want to be able to
grant other researches access to their storage spaces as well.

This pattern of data manager and principal investigator having very similar
stories continues again when we see that they both want to be able to request
more storage space for their projects. While most storage requests were in the
first priority group of user stories these ones are in the second priority
group because they are requesting additional storage which isn't a critical
feature.

The fourth story provides a good insight into the power of the principal
investigator as they want to be able to create data managers from researchers.
This functionality is important because it allows the system to be mostly run
by users in their own project rather than having the global administrators set
roles for each project which could become a large amount of work if the system
is used across a large user base.

There were two more user stories that were set as priority two.

\begin{itemize}
	\item As an administrator, I want to change the principal investigator
	      for a storage space to replace the existing principal
	      investigator; and
	\item As an approver, I want to view a list of all storage requests so
	      that I am aware of all requests by principal investigators in my
	      faculty.
\end{itemize}

From this first story we see that administrators need to have the power to
change principal investigators of storage spaces. This is important because if
a principal investigator doesn't do their job properly it could be damaging to
the project. This also helps us better define the role of the administrators by
giving more insight into their needs.

The second story helps us understand the position of approvers within the
company because of they want to be able to see faculty based requests we can
understand that approvers would be a high position within the faculty and need
to see what researchers within the faculty need. This also helps show us that
the user classes we create must have a faculty identifier somewhere so that
this story can be met.

The third and final group of user stories were the stories we decided were the
lowest priority to complete in the system. These were features and functions
that did not have much of an impact on the functionality but were more for ease
of use.

\begin{itemize}
	\item As a data manager, I want to revoke an existing researcher's
	      access to my storage space so that they can't access my data;
	\item As a principal investigator, I want to revoke an existing
	      researcher's manager level access for my storage space so that
	      they cannot submit administrative requests;
	\item As a principal investigator, I want to revoke existing
	      researchers' access to my storage space so that they can't access
	      my data; and
	\item As a researcher, I want to receive email notification when my
	      access to a storage space is revoked so that I am aware that I
	      can no longer access it.
\end{itemize}

There were quite a few user stories relating to revoking access of a researcher
either to a space or of their administrative powers of that space. We felt that
these would definitely not be needed to demonstrate the product in the few
sprints and that there would be no point doing before researchers could be
added.

These user stories are much straighter forward than the earlier stories because
they are relating to simple functions that help with the maintainability of the
software. There is not more to be read out of the first three stories other
than the different high level role users want to be able to remove researchers
from their projects. It is a simple request and not a hard feature to implement
which is why it was pushed back to the lowest priority.

The fourth user story in this set is a bit different though because it is from
the researcher who wants to be notified about being removed from the storage
space. Email notifications were something that we decided did not need to be
down until all of the core functionality is in and working properly. So all
email and notification stories were placed in to the lowest priority group.

\begin{itemize}
	\item As a data manager, I want to receive email notifications so that
	      I am aware that requests have been completed;
	\item As a principal investigator, I want to receive email
	      notifications so that I am aware that requests have been
	      completed; and
	\item As a researcher, I want to receive email notifications when I am
	      granted access to a storage spaces so that I can access the data.
\end{itemize}

All of these stories are about email notifications for features that were
requested in higher priority stories. From these though we see that the system
does need to be able to sync to a mail server to deliver the email and that all
users need to register an email with their account.

One of the notification requests however was different, `As an administrator, I
want to send appropriate notifications so that users can start using their
storage spaces.' From this story we need to create a way for administrators to
be able to send notifications to users. This story in particular did not
however mention what kind of notification should be sent so it was up to us to
decide how best to send a notification.

\begin{itemize}
	\item As an administrator, I want to view a list of requests by
	      submission date so that I can review them chronologically; and
	\item As an approver, I want to view a list of all storage spaces for
	      my faculty so that I can determine storage space allocated for my
	      faculty.
\end{itemize}

These two stories show that the high level users want to be able to list
submission requests and have the sorted in a particular way. This meant that
some sort of algorithm would need to be implemented to sort them; while the
first story also refers to submission date which none of the others do, this
may have not been stored otherwise. Although similar there are other stories
that make a note of faculty so it would be a bit more obvious to store which
faculty a project is part of.

The final user stories are about requests for storage.

\begin{itemize}
	\item As a data manager, I want to view the status of my request so
	      that I know its progress;
	\item As a principal investigator, I want to view the status of my
	      request so that I know its progress; and
	\item As an administrator I want to add a comment to a request so that
	      I can highlight any special requirements or additional
	      information related to the request.
\end{itemize}

These stories show that users want to be able to see the status of their
requests so they can then follow up with administrators personally if need be.
The third story here does ask for something a bit different though and that is
the ability to add comments to requests. This does require adding on some extra
functionality, however as a group we decided that it was not core functionality
and did not need to be put in until all of the core features and functions were
added.

For the product backlog we ended up with seven user stories being classified as
highest priority, these were the core functions and features of the system.
There were six user stories as medium priority which was mostly usability
features such as increased user control. The lowest priority group had thirteen
user stories although as explained above many of these were the same story from
different users and most of them were increased user control or notification
requests.

\section{Sprint documentation}

\subsection{Sprint 1 planning}

The user stories that were chosen for the first sprint were the user stories
with the greatest importance, serving as the metaphorical foundations of the
project. To do this we rearranged and categorised the user stories by their
importance, access level requirements, and dependencies to yield the tasks
which required the lowest level of requirements to complete first. The first
sprint should contain everything that would be mandatory in the second and
third sprints, namely:

\begin{itemize}
	\item As an administrator, I want to log in so that I can review and
	      approve requests;
	\item As a principal investigator, I want to submit storage requests so
	      that I can have access to a storage space for a research project;
	\item As a user, I want to log in so that I can perform appropriate
	      functions applicable to my role;
	\item As a user, I want to view a list of my storage spaces so that I
	      know my access level for my research projects;
	\item As an administrator, I want to approve a request so that storage
	      can be provisioned, expanded or access permissions changed;
	\item As an approver, I want to log in so that I can approve storage
	      requests; and
	\item As an approver, I want to review and approve a storage request so
	      that storage can be provisioned.
\end{itemize}

These user stories are what we collectively prioritised as the most important
tasks to complete before any other parts of the project can be undertaken.

The breakdown that we had planned for the project as a whole was to correlate
all user stories to a number depending on what required the least functionality
as well as which user stories required others to be completed before they could
be implemented. By doing this, we managed to organise all of the user stories
into three groups, which ideally would be allocated to each sprint. The first
sprint would involve everything on the lowest level, which includes logging
into the website, viewing spaces and reviewing requests as the users with the
highest levels of access permissions.

With this we estimated that it would take approximately two weeks. This would
give us enough time for the group to orientate and establish suitable
communications between all members. After that, we would be able to work
smoothly throughout the project. Within this fortnight, one week will be set
aside to create the user login system and have it operational, then three days
will be allocated to implement the viewing of lists and reviewing of storage
requests. The remaining four days would go towards fixing defects and polishing
what has been done thus far, in addition to working on other projects.

\subsection{Sprint 1 implementation}

Planned tasks:

\begin{itemize}
	\item Set up project management tools;
	\item Authentication system;
	\item Submit storage and additional storage requests;
	\item View a list of all users' and faculty storage spaces; and
	\item Approve and review requests.
\end{itemize}

Completed tasks:

\begin{itemize}
	\item Set up project management tools;
	\item Authentication system;
	\item Submit storage and additional storage requests;
	\item Redesign of navigation bar; and
	\item Frontend addition of interleaved links to relevant pages.
\end{itemize}

Due to most of the group being new to web development, we focused on achieving
the functionality of the user stories, and decided to adjust the frontend to
match the inital mockups at a later date.

Multiple accounts with no access restrictions were successfully registered
using the website, and all were able to submit storage requests as per the
requirements specification. We have been testing the website using the latest
released of Google Chrome, Mozilla Firefox and Internet Explorer. At this stage
the registration screen accepted a username and password, with no requirement
for Curtin ID or email address, therefore checking for a valid email address
was not required.

The task effort breakdown for the first sprint is illustrated in
Figure~\ref{fig:effort1}.

\subsection{Sprint 1 review}

In the first sprint, the user stories decided upon were fairly ambitious. As a
group we completed a number of allocated user stories including:

\begin{itemize}
	\item As an administrator, I want to log in so that I can review and
	      approve requests;
	\item As a user, I want to log in so that I can perform appropriate
	      functions applicable to my role; and
	\item As an approver, I want to log in so that I can approve storage
	      requests.
\end{itemize}

These user stories overlapped in functionality and so were considered as one
single task, which was completed by Delan.

\begin{itemize}
	\item As a principal investigator, I want to submit storage requests so
	      that I can have access to storage space for research projects;
	\item As a data manager, I want to submit additional storage requests
	      so that I can store more research data in my storage space; and
	\item As a principal investigator, I want to submit additional storage
	      requests so that I can store more research data in my storage
	      space.
\end{itemize}

As with the login user stories, the above stories were considered to be one
single task due to the similarity in their functionality, and they were
completed by Kye.

\begin{itemize}
	\item As a user, I want to view a list of my storage spaces so that I
	      know my access level for my research projects; and
	\item As an approver, I want to view a list of all storage spaces for
	      my faculty so that I can determine storage space allocated to my
	      faculty.
\end{itemize}

This task was not completed in its entirety at this stage but the majority of
it had been completed and was ready for presentation. Completion of this task
was a joint effort between Jason and Kye.

There were also planned user stories that at this point were not yet complete,
such as:

\begin{itemize}
	\item As an administrator, I want to approve a request so that storage
	      can be provisioned, expanded or access permissions changed; and
	\item As an approver, I want to review and approve a storage request so
	      that storage can be provisioned.
\end{itemize}

Unfortunately, due to a combination of insufficient time management, informal
role allocation and a general team-wide unfamiliarity with the chosen
development environment, these user stories were not addressed.

Additional items completed concerning improvements to user experience include:

\begin{itemize}
	\item Redesign of navigation bar; and
	\item Front end addition of interleaved links to relevant pages.
\end{itemize}

These changes were completed by Luke.

\subsection{Sprint 1 retrospective}

This sprint was objectively rather productive. Of ten assigned tasks, the group
worked together to complete eight requirements that were initially planned, as
well as the addition of two unplanned tasks. These additions were vital in
order to ensure easy navigation of the project by the end user.

Issues encountered mainly concerned:

\begin{itemize}
	\item The lack of a clear definition of group member task allocations.
	      This was the most fundamental flaw with our approach to this
	      sprint. Due to a combination of poor communication and job
	      allocation, we had a situation where many group members had
	      accidentally begun work on the same tasks and for this reason,
	      tasks were left untouched while a lot of effort was duplicated
	      and ultimately discarded.
	\item Many team members at this stage were completely unfamiliar with
	      the chosen working environment of ASP.NET MVC 5.
	      While not quite as important as the previous issue, at this
	      stage, it was definitely a hindrance to be trying to write code
	      in an environment that none of the group members were familiar
	      with.
\end{itemize}

For these reasons, the group as a whole resolved to be more explicit in task
allocations and ownership. Expected task time frames were decided on moving
forward, as a preventative measure for tasks being left incomplete.

\subsection{Sprint 2 planning}

What was designated for the second sprint were the second set of user stories,
which involved building up from what should have been accomplished in the first
sprint. Ideally, this would mean that authentication and request approvals
would have been completed.

The second set of user stories would involve creating a access permissions
system whereby --- depending on a particular user's access level --- the system
would provide and/or restrict functionality. The user stories that were chosen
for the second sprint include:

\begin{itemize}
	\item As a data manager, I want to grant other researchers access to my
	      storage space so that I can collaborate with others;
	\item As a data manager, I want to submit additional storage requests
	      so that I can store more research data in my storage space;
	\item As a principal investigator, I want to grant other researchers
	      access to my storage space so that I can collaborate with others;
	\item As a principal investigator, I want to grant other researchers
	      manager level access for my storage space, so that they can
	      submit administrative requests;
	\item As a principal investigator, I want to submit additional storage
	      requests so that I can store more research data in my storage
	      space;
	\item As an administrator, I want to change the principal investigator
	      for a storage space to replace the existing principal
	      investigator; and
	\item As an approver, I want to view a list of all storage requests so
	      that I am aware of all requests by principal investigators in my
	      faculty.
\end{itemize}

Assuming the aforementioned user stories are completed by the end of the
sprint, the project should also be able to handle the next set of user stories
to be implemented. This is because the second sprint is purely intended to
ensure that access levels and data storage function correctly.

From the user stories outlined above, the second level of the breakdown is
found, including the tasks that require logging in as a user with a specific
user access level. This involves submitting and granting storage space requests
for `middle level' users.

The time estimated for this sprint would again be two weeks, with the granting
of storage requests taking one week. The additions to the prototype that handle
request submission should only take three days at most to implement, and the
remaining days would be allocated to the ability to view and change the
principal investigator of storage spaces.

\subsection{Sprint 2 implementation}

Planned tasks:

\begin{itemize}
	\item View a list of all storage requests;
	\item Grant other researchers access to my storage spaces;
	\item Approve and review storage requests;
	\item User access levels and privileges; and
	\item UML diagram.
\end{itemize}

Completed tasks:

\begin{itemize}
	\item Dashboard frontend with static HTML;
	\item Cancellation and deletion of space requests;
	\item Recorded timestamps for each request; and
	\item UML diagram.
\end{itemize}

In an earlier meeting Callum suggested using the Decorator software pattern to
implement the different user roles as they have subset permissions. The group
agreed to research the Decorator pattern in order to use it. However, due to
the complexity of the MVC5 framework and our lack of experience using it, it
was decided that the pattern was too complex to implement using a foreign
framework.

We then decided our account level types will be implemented as different
viewing permissions. This allows us to base what the user is allowed to see on
their role as per the requirement specification. Although, the tradeoff is that
the website is not as secure as would be preferred. Due to the time constraints
the group decided to pursue the safer option of fulfilling the required
functionality with lesser security measures.

The task effort breakdown for the second sprint is illustrated in
Figure~\ref{fig:effort2}.

\subsection{Sprint 2 review}

At the epoch of the second sprint, user stories decided on were scaled back
quite drastically to accommodate incomplete items in the sprint backlog. Of the
chosen user stories, the following were completed:

At this stage in the project, there were a worrying number of incomplete user
stories now moved into our unfinished item backlog, such as:

\begin{itemize}
	\item As an approver, I want to review and approve a storage request so
	      that storage can be provisioned;
	\item As an approver, I want to view a list of all storage requests so
	      that I am aware of all requests by principal investigators in my
	      faculty;
	\item As a data manager, I want to grant other researchers access to my
	      storage space so that I can collaborate with others;
	\item As an administrator, I want to approve a request so that storage
	      can be provisioned, expanded or access permissions changed; and
	\item As an approver, I want to review and approve a storage request so
	      that storage can be provisioned.
\end{itemize}

Of these, the most important functionality by far were the user access levels,
and to not yet have this complete took a very noticeable cleave out of the
team's morale.

In addition to the planned functionality to be worked on in this sprint, there
were also a number of features added supplementary to what was planned:

\begin{itemize}
	\item Dashboard front end using static HTML.
\end{itemize}

As completed by Jasmine, this was an invaluable feature to have going into the
presentation for the second sprint as it was the first major step towards
transposing our initial mockups and planning into an actual product.

\begin{itemize}
	\item Users being able to cancel/delete their own space requests; and
	\item Requests now storing a timestamp to reflect when they were made.
\end{itemize}

These functionalities were implemented by Kye.

\subsection{Sprint 2 retrospective}

Aspects that worked well at this point were unfortunately rather scarce.

Over the course of the sprint, many obstacles became apparent:

\begin{itemize}
	\item Members were sick due to the time of the year. While it is
	      unavoidable that this can happen, the team unfortunately had made
	      no provisions to handle the situation. When people were unable to
	      complete their tasks, there was little and rather poor
	      communication within the group to attempt to overcome the issue.
	\item Sprint backlog items were not adequately addressed. Over the
	      course of this sprint, the team morale reached a global minimum.
	      Consequently team members felt uncompelled to address any tasks
	      which resulted in a less than desirable yield.
\end{itemize}

After what was undeniably the worst sprint so far, the team as a whole devised
stricter guidelines for the next sprint in order to produce more consistent
results going forward. Guidelines considered of utmost importance were:

\begin{itemize}
	\item Motivate team members more effectively to stick to designated
	      tasks and time schedules; and
	\item Issues that arise will be brought up both in person as well as
	      the Facebook group in order to alert all group members of
	      potential problems as soon as they develop.
\end{itemize}

\subsection{Sprint 3 planning}

By the third sprint hopefully, the previous user stories would have been
completed and implemented correctly. The rest of the user stories are small
additions onto the already existing system, so even though there are a vast
quantity more user stories in the sprint than others, these user stories all
function in a similar fashion to one another.

The user stories selected to be implemented in the third sprint were:

\begin{itemize}
	\item As a data manager, I want to receive email notifications so that
	      I am aware that requests have been completed;
	\item As a data manager, I want to revoke existing researchers' access
	      to my storage space so that they can't access my data;
	\item As a data manager, I want to view the status of my request so
	      that I know its progress;
	\item As a principal investigator, I want to receive email
	      notifications so that I am aware that requests have been
	      completed;
	\item As a principal investigator, I want to revoke an existing
	      researcher's manager level access for my storage space so that
	      they cannot submit administrative requests;
	\item As a principal investigator, I want to revoke existing
	      researchers' access to my storage space so that they can't access
	      my data;
	\item As a principal investigator, I want to view the status of my
	      request so that I know its progress;
	\item As a researcher, I want to receive email notification when I am
	      granted access to a storage space so that I can access the data;
	\item As a researcher, I want to receive email notification when my
	      access to a storage space is revoked so that I am aware I can no
	      longer access it;
	\item As an administrator, I want to add a comment to a requests so
	      that I can highlight any special requirements or additional
	      information related to the request;
	\item As an administrator, I want to send appropriate notifications so
	      that users can start using their storage spaces;
	\item As an administrator, I want to view a list of requests by
	      submission date so that I can review them chronologically; and
	\item As an approver, I want to view a list of all storage spaces for
	      my faculty so that I can determine storage space allocated for my
	      faculty.
\end{itemize}

The time available would be three weeks, and this would include finishing the
formatting and patching all remaining defects, as well as ensuring that the
system runs in accordance with the client's requirements.

The email notification module would require three days, and viewing the status
of requests would require a further four days to implement. The second week
would be dedicated to revocation of storage space access, as this involves
changing the access control data between a user and a space, rather than the
access role of a user.

\subsection{Sprint 3 implementation}

Planned tasks:

\begin{itemize}
	\item Viewing a list of all storage requests;
	\item Approve and review storage requests;
	\item User access levels and privileges;
	\item Granting access to storage spaces;
	\item Granting manager level access to storage spaces;
	\item Sending appropriate notifications;
	\item Receiving email notifications;
	\item Viewing status of requests;
	\item Revoking access to storage spaces; and
	\item Adding comments to requests.
\end{itemize}

Completed tasks:

\begin{itemize}
	\item Viewing a list of all storage requests;
	\item Approve and review storage requests;
	\item User access levels and privileges;
	\item Granting access to storage spaces;
	\item Granting manager level access to storage spaces;
	\item Sending appropriate notifications;
	\item Receiving email notifications;
	\item Viewing status of requests;
	\item Revoking access to storage spaces; and
	\item Adding comments to requests.
\end{itemize}

As the project is not being hosted online, we are unable to actually send
email notifications. Instead, we are working around this limitation with a
page that appears when an email notification would have been sent that will
still have the same content and data.

The task effort breakdown for the third sprint is illustrated in
Figure~\ref{fig:effort3}.

\subsection{Sprint 3 review}

It was near the end of the second sprint where the team didn't realise how
close the deadline was to the final presentation of the project, and it was
only then did the group finally shift gears and come together to start
finishing off the user stories rather quickly. With efforts from Kye and Luke,
the user access levels were quickly finished and with that, the following list
of user stories were also completed with it, from the backlog of sprint two:

\begin{itemize}
	\item As an approver, I want to review and approve a storage request so
	      that storage can be provisioned;
	\item As an approver, I want to view a list of all storage requests so
	      that I am aware of all requests by principal investigators in my
	      faculty;
	\item As a data manager, I want to grant other researchers access to my
	      storage spaces so that I can collaborate with others;
	\item As an administrator, I want to approve a request so that storage
	      can be provisioned, expanded or access permissions changed; and
	\item As an approver, I want to review and approve a storage request so
	      that storage can be provisioned.
\end{itemize}

When this was completed, Jason in addition to Luke and Kye worked tirelessly to
get the project back onto a reasonable pace, and could possibly complete it on
time. These were the remaining user stories that required work:

\begin{itemize}
	\item As a data manager, I want to revoke existing researchers' access
	      to my storage spaces so that they can't access my data;
	\item As a data manager, I want to view the status of my request so
	      that I know its progress;
	\item As a principal investigator, I want to receive email
	      notifications so that I am aware that requests have been
	      completed;
	\item As a principal investigator, I want to revoke an existing
	      researcher's manager level access for my storage spaces so that
	      they cannot submit administrative requests;
	\item As a principal investigator, I want to revoke existing
	      researchers' access to my storage spaces so that they can't
	      access my data;
	\item As a principal investigator, I want to view the status of my
	      requests so that I know its progress;
	\item As an administrator, I want to send appropriate notifications so
	      that users can start using their storage spaces;
	\item As an administrator, I want to view a list of requests by
	      submission date so that I can review them chronologically; and
	\item As an approver, I want to view a list of all storage spaces for
	      my faculty so that I can determine storage space allocated for
	      my faculty.
\end{itemize}

\subsection{Sprint 3 retrospective}

Our team eventually managed to coordinate efforts relatively smoothly,
finishing the prototype --- and more broadly, the project --- on time, although
the stage at which this occurred was significantly later than what would be
optimal.

The vast majority of the user stories provided by the client were implemented
in this final sprint, in an effort to count the subpar productivity yielded by
the former two sprints. By this time in the semester however, the ability of
group members to find enough time to complete the shared workload became
significantly more difficult, which bolstered the fragility of our commitment
to the project deadline.

\subsection{Details of the design}

After much deliberation over the user stories and the permissions model, our
team devised a set of models and their relationships as defined by the UML and
entity-relationship diagrams in Figure~\ref{fig:uml} and Figure~\ref{fig:erd}
respectively.

It is apparent that there is much in common between the model classes
\texttt{NewSpaceRequest} and \texttt{IncreaseSpaceRequest}. While
we initially tried to refactor this by making them subclasses of an abstract
parent class \texttt{Request}, this conflicted with the
\texttt{EntityFramework} object-relational mapper that is an integral part of
ASP.NET MVC 5. The ORM would return instances of the parent class, which would
require casting, then fail to store in the database when changed. This
impasse forced us to allow the small amount of code duplication required
to eliminate the inheritance relationship in our mapped classes.

\figimg{uml}{uml}{The Unified Modeling Language diagram for this prototype}

\figimg{erd}{erd}{The entity-relationship diagram for this prototype}

\begin{landscape}
	\quad % ugly hack because there needs to be some text here
	\thispagestyle{empty}
	\begin{textblock}{270}(-12,0)
		\figimg[270mm]{effort1}{effort1}
			{The task effort chart for the first sprint}
	\end{textblock}
\end{landscape}

\begin{landscape}
	\quad % ugly hack because there needs to be some text here
	\thispagestyle{empty}
	\begin{textblock}{270}(-12,0)
		\figimg[270mm]{effort2}{effort2}
			{The task effort chart for the second sprint}
	\end{textblock}
\end{landscape}

\begin{landscape}
	\quad % ugly hack because there needs to be some text here
	\thispagestyle{empty}
	\begin{textblock}{270}(-12,0)
		\figimg[270mm]{effort3}{effort3}
			{The task effort chart for the third sprint}
	\end{textblock}
\end{landscape}

\subsection{Minutes for Scrum meeting 1}

Date: 2014-04-03

Start time: 4:00 p.m.

Discussing how the project should be completed and the technologies we are
using to implement the prototype.

\begin{itemize}
	\item Front end: Bootstrap;
	\item Back end: ASP.NET MVC 5;
	\item Database: Microsoft SQL Server;
	\item Time logging: Google Docs form;
	\item Number of sprints: three.
\end{itemize}

Front end tasks to complete during the first sprint:

\begin{itemize}
	\item Requests pages and forms;
	\item Login screen; and
	\item Notification space.
\end{itemize}

End time: 5:00 p.m.

\subsection{Minutes for Scrum meeting 2}

Date: 2014-04-15

Start time: 12:10 p.m.

Meeting for discussing what is functioning in the first sprint, and what needs
to be added during sprint two.

The list of user stories that were completed were:

\begin{itemize}
	\item Submission of storage requests; and
	\item Users being able to log in.
\end{itemize}

Tasks that were partially completed were:

\begin{itemize}
	\item Administrators being able to approve and change permissions; and
	\item Principal investigators submitting storage requests for research.
\end{itemize}

All of the user stories were completed on the front end load where users could
see what it looked like, but all of the functionality was yet to be finished.
These will need to be completed within the next sprint because there were
issues with access levels of the different types of users, and we were unsure
how to handle them until we completely designed the access levels for all the
users, which would be done in the second sprint.

End time: 1:00 p.m.

\subsection{Minutes for Scrum meeting 3}

Date: 2014-04-25

Start time: 1:03 p.m.

Task allocation:

\begin{itemize}
	\item Robert: UML diagram, user stories, list of back end and front end
	      design decisions, implementation and sprint review;
	\item Jasmine: high level task breakdown, effort estimation;
	\item Callum: implementation of approvers' user stories;
	\item Delan and Luke: implementation of granting and revocation of
	      permissions for data managers;
	\item Kye: implementation of user authentication and access levels; and
	\item Jason: implementation of the permissions list view.
\end{itemize}

All further issues will be raised on the Facebook group, so that all members
can have the opportunity to assist. The original plan from earlier sprints was
too complicated, and it will be modified to make it simpler. Approvers will be
managed with a flag at the user level, indicating whether they are allowed to
approve requests or not.

The UML diagram was designed for the implementation.

End time: 2:34 p.m.

\subsection{Minutes for Scrum meeting 4}

Date: 2014-05-01

Start time: 4:00 p.m.

Meeting for discussing what is functioning in sprint two and what needs to be
added to sprint three, as well as allocation of report writing tasks to group
members. The user stories that were planned for full implementation were for
data managers and principal investigators to be able to submit requests, and to
allow privileged users to grant and revoke access to storage spaces.

The list of user stories that were completed were:

\begin{itemize}
	\item Submission of storage requests for data managers and principal
	      investigators;
	\item Viewing a list of storage spaces for approvers;
	\item Administrators being able to approve and change permissions; and
	\item Principal investigators submitting storage requests for research.
\end{itemize}

User stories that were partially completed were:

\begin{itemize}
	\item Administrators being able to change the principal investigator on
	      a storage space; and
	\item Principal investigators granting researchers read, write and data
	      manager level access to storage spaces.
\end{itemize}

As a team, the decision was made to shift a majority of the focus towards the
documentation of the sprints and the report. The report will be created for
global editing within the members to update and edit where possible. We will
finish the demonstration for displaying research by the next week. Jasmine
has been assigned to conduct the final presentation. Callum and Robert have
been assigned to write the `details of design' and `Scrum meeting' sections of
the final report. Luke and Delan were assigned to completing the backend models
and controllers, while Jason and Kye are to finish developing the views and the
user interface. Callum shall polish the finished project for presentation and
any additional documentation within the report.

End time: 4:37 p.m.

\subsection{Minutes for Scrum meeting 5}

Date: 2014-05-15

Start time: 3:59 p.m.

By the end of sprint three, we had assumed that everything for the project
would be functional, except for some bugs which are to be expected.

The list of user stories that were completed were:

\begin{itemize}
	\item Administrators being able to change the principal investigator
	      for a storage space;
	\item Principal investigators granting researchers read, write and data
	      manager level access to storage spaces;
	\item Reception of email notifications as any privileged user;
	\item Revocation of access as any user;
	\item Addition of comments to requests;
	\item Viewing a list of request sorted by submission date; and
	\item Viewing a list of storage spaces for a specific faculty.
\end{itemize}

Tasks that were partially completed were:

\begin{itemize}
	\item Formatting for the user interface.
\end{itemize}

Report task reallocation:

\begin{itemize}
	\item Callum: product backlog;
	\item Delan: details of design, conclusion and summary;
	\item Jasmine: background, implementation details;
	\item Jason: abstract, introduction and objective;
	\item Kye: project review;
	\item Luke: sprint review and retrospective; and
	\item Robert: sprint documentation and planning.
\end{itemize}

Callum and Jasmine will edit and correct the final report copy, while Delan
will complete the formatting and typesetting of the report.

End time: 4:24 p.m.

\subsection{Minutes for Scrum meeting 6}

Date: 2014-05-26

Start time: 9:11 a.m.

Finalisation of report, documentation and presentation.

The group shall have the report finished by Friday for editing and formatting.
Delan will create the PowerPoint file for the final presentation. Depending on
Aneesh's answer, everyone could be presenting, or only some group members.

End time: 9:21 a.m.

\subsection{Minutes for Scrum meeting 7}

Date: 2014-05-29

Start time: 4:06 p.m.

Last meeting before the final presentation.

Contact Delan when finishing report segments so formatting can be completed.
Jasmine is unwell; if she is unable to present, duties will be passed onto
Callum, Robert or Luke depending on the day of the presentation. Small changes
will be made to the prototype.

End time: 4:27 p.m.

\section{Project review}

There were many lessons learned from working on this project using the Scrum
process. We did quite a few things right with our process. In our initial group
meetings we sorted out our product backlog right away which really gave us a
good idea of what we needed to do. We also split the work load up into the
three sprints and set our sprint goals. Our meeting organisation was quite well
done given the size of the group and the available time everyone had given
their respective units, we were always able to have weekly meetings even over
the break.

Initially the group was quite motivated and quite eager to get started however
the exact work of each group member was not clearly defined and the team was
not communicating as much as they need to be for using a process like Scrum.

While the group was meeting weekly, these meetings were often missing a couple
of members, it was not having daily Scrum meetings and the group was not
communicating about the project to each other outside of the weekly meetings.
This led to a lot of work being done unnecessarily. Another major problem
facing the group was the unfamiliarity with ASP.NET MVC, only one member had
previously used it and the rest were learning as they were going which led to
some members work being of an unacceptable low quality.

All of these issues led to the group becoming very unmotivated for the time of
the second sprint. This was also the same time other units had begun to really
require time for their assignments. During this time communication was at an
all-time low and the members did not know what members were doing. During this
time very little work was done.

Another major problem was the lack of a Scrum master for the group. Without a
Scrum master the group did not really know how to follow the process and it
resulted in work not being done and time for work not being logged. If we had
closely followed Scrum like we were supposed to many of our problems would not
have occurred however no-one in our group had a very good understanding of
Scrum.

Throughout this project we learned quite a few lessons. The most important is
having a smaller group, with a process like Scrum more people only makes it
harder and in a university setting trying to have a meeting with 7 people is
quite hard. We learned that we need to not get unmotivated as easily otherwise
it results in a lot more stress and rushed work down the line.

For future students we would definitely recommend maintaining a small group
rather than thinking a large group would mean less work overall. The amount of
work in taking to work well together as a team increases a lot for each new
person added in to the project.

\section{Conclusions and summary}

In the last three months, our team \textit{To the Moon} has conceived,
designed, developed, tested and delivered a functional prototype for a storage
management system that could be used in a university setting. We strived to
adopt as many industry standard methodologies, paradigms and technologies as
we could, including the Scrum method, the model-view-controller pattern and
distributed version control.

Many factors introduced tangible challenges against the efficacy of our
teamwork, such as the coordination of communication amongst a diverse team of
seven members, a hostile time environment where this project must be balanced
with other unrelated assessements and an unavoidable unfamiliarity with some of
the technologies involved.

The overall problem that needed to be solved was the scalability of a software
development team. Of course, no team in existence scales in a purely linear
manner against the number of members, but it is an ideal that we must all
strive for, as non-trivial software projects that can be developed completely
by an individual are scarce.

For a significant portion of this group's members, \textit{Project Design and
Management 300} served as our first group software project, serving as a marked
change from histories of individual hobby projects and educational assessments.
As such it is understandable that the project did not flourish as smoothly as
we would have hoped.

To round this report off, the following are a summary of the lessons learned:

\begin{itemize}
	\item Strive to organise \textit{regular} meetings that include
	      \textit{all} group members, whether or not they are
	      conducted in person or online;
	\item Allocate roles formally and strictly to minimise duplication of
	      effort;
	\item Place a greater emphasis on working with technologies that the
	      most group members are familiar with, when opposed to ideological
	      purity;
	\item Make clear and realistic provisions to handle the illness of team
	      members;
	\item Ensure that developers remain constantly motivated throughout the
	      project; and
	\item Clarify any ambiguous components of user stories and other client
	      documentation as soon as possible.
\end{itemize}

\begin{sloppypar}
	\printbibliography
\end{sloppypar}

\end{document}
